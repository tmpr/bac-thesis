\documentclass{article}

% -- Packages
\usepackage[utf8]{inputenc} \usepackage{biblatex} \usepackage{hyperref}
\usepackage{parskip} \usepackage{amsmath} \usepackage{amssymb}
\usepackage{microtype}

% -- Theorem environments
\newtheorem{definition}{Definition} \newtheorem{example}{Example}

% -- Document Settings

\emergencystretch=1em

\title{Deriving BLOSUM-matrices for protein-coding DNA}
\addbibresource{substitution-matrices.bib} \author{Alexander Temper}

\begin{document}
\maketitle

\begin{abstract}
    Sequence Alignment algorithms rely on some form of substitution matrix,
    which can heavily influence their results. In this work, we try to derive a
    substitution matrix for the genes (DNA) of a given protein family akin to
    the BLOSUMx matrices, which are used for amino acid sequences.
\end{abstract}

\tableofcontents

\section{Introduction} Test. Sequence alignment is the task of pairing up the
elements of two (biological) sequences with the aim of finding conserved
regions and homologous sequences which are anticipated to have some
evolutionary relation. Sequence alignment algorithms are primarily used for
constructing phylogenetic trees, searching for homologous sequences with BLAST,
assembling full DNA strands from short reads, creating input features for
AlphaFold2 and other.
% How formal should we introduce the problem.
Most alignment algorithms try to maximize a so called \emph{scoring function},
which assigns to a given alignment some score. The classical methods score
alignments by summing up scores assigned to each pair of letters in the given
alignment. The scores for the pairs of letters can be represented by a
symmetric matrix \- we call such matrices \emph{scoring matrices}. Next, we
formally define our vocabulary. W.l.o.g., let us restrict ourselves to pairwise
alignments, i.e., alignments of two sequences.

\begin{definition}
    The following will be used throughout the paper:
    \begin{itemize}
        \item A \emph{sequence} $s$ is an ordered collection of letters from
              some alphabet of letters $\mathcal{L}$.
        \item The letter '\texttt{-}' is called \emph{indel} and represents
              gaps in the alignment, which occur due to mutations or sequencing
              errors.
        \item A (pairwise) \emph{sequence alignment} is a matrix $\mathbf A \in
                  (\mathcal{L} + \{\mathtt{-}\})^{2 \times m}$, where $m$ is the
              length
              of the alignment. We symbolize the set of all alignments of
              alphabet $\mathcal L$ of length $m$
        \item A scoring function $\sigma: (\mathcal{L} + \{\mathtt{-}\})^{2
                      \times m} \to \mathbb{Z}$ maps an alignment to its score.
    \end{itemize}
\end{definition}
Most algorithms of concern here use a scoring function of the form
$\sigma(\mathbf A) = \sum_{j=0}^m s(\mathbf A_{1j}, \mathbf A_{2j})$, where $s:
    (\mathcal L + \{\mathtt{-}\})^2 \to \mathbf{Z}$ is a symmetric function
evaluating pairs of letters, which can be represented by a matrix $\mathbf S
    \in \mathbb Z^{(\# \mathcal L) + 1 \times (\# \mathcal L) + 1}$. Said matrix is
the focus of this work.

\begin{example}
    Since we are talking about DNA, our alphabet of concern will be the 4
    different nucleotide bases, \{\texttt{A, C, G, T}\}. Two exemplary DNA
    sequences are \texttt{ACA} and \texttt{AAGA}. One possible alignment between them is
    \begin{equation*}
        \mathbf A = \begin{bmatrix}
            \mathtt{A} & \mathtt{C} & \mathtt{-} & \mathtt{A} \\
            \mathtt{A} & \mathtt{A} & \mathtt{G} & \mathtt{A}
        \end{bmatrix}
    \end{equation*}
    Indexes where the two nucleotides are equal are called \emph{matches},
    where the two
    are not equal are called \emph{mismatches} and where an indel is matched to
    a nucleotide are called \emph{gaps}. An exemplary scoring matrix might be
    \begin{equation*}
        \mathbf S = \begin{matrix}
             & \mathtt{A} & \mathtt{C} & \mathtt{G} & \mathtt{T} &
            \mathtt{-}                                             \\ \mathtt{A} & 1          & -2         & -3         & 0
             &
            0
            \\ \mathtt{C} & -2         & 0          & 0          & 0
             & 0
            \\ \mathtt{G} & -3         & 0          & 0          & 0
             & -4
            \\ \mathtt{T} & 0          & 0          & 0          & 0
             & 0
            \\ \mathtt{-} & 0          & 0          & -4         & 0
             & 0
            \\
        \end{matrix}
    \end{equation*}
    Using the above matrix, \begin{equation*}
        \sigma(\mathbf A) = s(\mathtt A, \mathtt A) + s(\mathtt C, \mathtt A) + s(\mathtt{-}, \mathtt{G}) + s(\mathtt{A}, \mathtt{A}) = 1 - 2 -4 + 1 = -6.
    \end{equation*}
\end{example}
\section{Previous work}

Most notably, the two families of classical scoring matrices are the PAM
matrices and the BLOSUM matrices. However, and this is also true for most other
research around sequence alignment: there is a strong emphasis on studying the
alignment of proteins over studying the alignment of DNA. To the best of our
knowledge, nearly all benchmarks concern \emph{only protein alignments}, which
we suspect to to be one of the reasons that there is limited research focusing
on DNA alignment. There have been attempts to create gold standards for RNA
alignment,
however, even there, investigations suggest an overrepresentation of tRNA, thus
leading to a suboptimal benchmark.

The literature for DNA scoring matrices derived from data is sparse.
\textcite{hamadaTrainingAlignmentParameters2017} derive scoring matrices from
specific organisms sequenced by specific sequencers. Further, as they claim,
different genes in different organisms have significantly differing rates of
mutation, which is why general-purpose DNA scoring matrices might be a bad
idea. Their main focus lies on recovering and mitigating the sequencing errors
and projecting the differing GC contents of different species in the resulting
matrix. They evaluate their data on simulated data, which by definition makes
some assumptions, thus is not too optimal. Their method performs slightly
better than 2 manually created scoring matrices.

\section{BLOSUM} As a successor to the PAM matrices,
\textcite{henikoffAminoAcidSubstitution1992} first constructed the BLOSUM
matrices for protein alignments. A wonderful explanation thereof was written by
\textcite{eddyWhereDidBLOSUM622004}, however, we shall briefly dive into the
theoretical underpinnings of BLOSUM here as well.

Underlying BLOSUM is the equation that, given the two paired letters $a \in
    \mathcal L, b \in \mathcal L$,
\begin{equation*}
    s(a, b)= \lambda \log_2 \frac{P(a, b)}{P(a)P(b)}.
\end{equation*}
Let us disect this:
\begin{itemize}
    \item $s(a, b)$ is the score of $a$ and $b$ being aligned.
    \item Our two hypotheses are are that
          \begin{enumerate}
              \item $a$ and $b$ are related evolutionarily and ought to be
                    aligned and
              \item $a$ and $b$ are aligned due to random chance.
          \end{enumerate}
    \item We are interested in the odds of the former being true over the
          latter and encoding this into a score.
    \item We can approximate this with already existing, aligned data. The
          original paper used the so called Blocks matrix for this.
    \item We can get approximate $P(a,b)$ by counting all the aligned pairs of
          the already existing alignment and normalizing those counts to
          probabilities. $P(a)$ can be computed analogously.
    \item For numerical reasons, we would like the score to be an integer,
          which is why we scale all all scores with $\lambda = 2$.
\end{itemize}
This family of matrices has become a de-facto standard for aligning amino acid
sequences - however, not so much for DNA. BLASTn, i.e., BLAST for nucleotide bases, currently uses a matrix where
matches are rewarded with +2 and mismatches are penalized with -3. It is this very assumed generality that has motivated this paper.
Interestingly enough, there have been
mistakes in the original computation of the BLOSUM matrices, which are claimed
to have been improving them quietly. This, however, might be attributed to the
fact that the benchmarks today might be influenced from the BLOSUM of the past.


\section*{Experiments and results}
\subsection*{Method}
We constructed a fully automated pipeline which takes as input an identification code of a protein
family on InterPro and the similarity threshold $x$ and computes the
corresponding BLOSUMx matrix. The source code can be found online, yet, here we
give some detail on the implementation and issues we faced.

First, we search the NCBI protein database for the given identification code
using eDirect. This yields only a subset of the desired genes, since not every
protein in the database is annotated with all protein family codes. It is
however, to the best of our knowledge, the fastest and most reliable way to
download genes of a given protein family on InterPro. For the scope of this
work, we deem this sufficient.

Next, the downloaded genes are preprocessed. Genes which contain elements
besides \texttt{ACGT} are removed. Then, the genes are being sorted into bins,
depending on their length, the rationale thereof being that most multiple
sequence alignments assume that the sequences are of similar length. Further,
bins which have too few sequences in them are also being removed.

Unfortunately, the PROTOMAT system originally used by Heinikoff is not to be
found online any longer. Thus, we create our own blocks, which is clearly a
major difference to the construction of the orgininal BLOSUM matrices.

NOTE: BLAST should provide Web interface for Matrix.
To do so, each bin is getting aligned by kalign (Use different software?) with
its default settings. This results in an MSA for each bin.

Thereafter, each MSA is looked at and columns which contain less than a certain
percentage of gaps are selected. Then, from the `dense' columns, we find blocks
of adjacent dense columns. These will be the blocks for the computation of the
BLOSUMx matrix. This is a somewhat

substantial difference: the blocks in the original paper are \textbf{gapless},
whereas the blocks in this paper \textbf{contain some gaps}.

\printbibliography
\end{document}

